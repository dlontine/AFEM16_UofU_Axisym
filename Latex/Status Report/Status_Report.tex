\documentclass[10pt,letterpaper]{report}
\usepackage[latin1]{inputenc}
\usepackage{amsmath}
\usepackage{amsfonts}
\usepackage{amssymb}
\usepackage{graphicx}
\usepackage[left=1.00in, right=1.00in, top=1.00in, bottom=1.00in]{geometry}
\author{Derek Lontine, Stuart Childs, Alex Bailey}
\title{AFEM Project: Axisymmetry}
\begin{document}
\maketitle
Our project is progressing well. We have been successful in creating and implementing axisymmetric elements (thanks to your help). We are currently in the process of determining if our implementation of the elements is correct. 

In the project assignment details, it is outlined that a project must complete two of the three following items:
\begin{enumerate}
\item Computational implementation
\item Analytical formulation
\item Commercial or production code use
\end{enumerate} 
In keeping with these items, we've determined that we would like to focus on the computational implementation and the analytical formulation. 

In addition to this, we are exerting a great deal of effort toward the verification and error analysis portion of the project. We feel that this portion of our project is most likely to produce re-usable results. We hope that the tests we create will help future developers in their development of axisymmetric elements in pyfem2.

Therefore, we have determined that the following outline captures what we aim to accomplish (or what has already been accomplished) in the project.

\bf{Outline:}\normalfont
\begin{itemize}
\item{Introduction}
\item{Progress on element implementation}
\subitem{Full integration element}
\subitem{Reduced integration with hourglass control}
\subitem{Selectively reduced integration}
\item{Brief description of verification problems we hope to verify against}
\item{How to evaluate each element type}
\end{itemize}
\end{document}