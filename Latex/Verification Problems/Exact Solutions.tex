\documentclass[10pt,letterpaper]{report}
\usepackage[latin1]{inputenc}
\usepackage{amsmath}
\usepackage{amsfonts}
\usepackage{amssymb}
\usepackage{graphicx}
\usepackage{blindtext}
\usepackage{listings}
\usepackage{enumerate}
\usepackage{enumitem}
\usepackage{hyperref}
\usepackage{cancel}
\usepackage{color,soul}
\usepackage{multirow}
\usepackage{float}
\usepackage[left=1.00in, right=1.00in, top=1.00in, bottom=1.00in]{geometry}
\author{Derek Lontine, Stuart Childs, Alex Bailey}
\title{AFEM: Axisymmetric Project Verification Tests}

\usepackage{color}
 
\definecolor{codegreen}{rgb}{0,0.6,0}
\definecolor{codegray}{rgb}{0.5,0.5,0.5}
\definecolor{codepurple}{rgb}{0.58,0,0.82}
\definecolor{backcolour}{rgb}{0.95,0.95,0.92}
 
\lstdefinestyle{mystyle}{
    backgroundcolor=\color{backcolour},   
    commentstyle=\color{codegreen},
    keywordstyle=\color{magenta},
    numberstyle=\tiny\color{codegray},
    stringstyle=\color{codepurple},
    basicstyle=\footnotesize,
    breakatwhitespace=false,         
    breaklines=true,                 
    captionpos=b,                    
    keepspaces=true,                 
    numbers=left,                    
    numbersep=5pt,                  
    showspaces=false,                
    showstringspaces=false,
    showtabs=false,                  
    tabsize=2
}
\lstset{style=mystyle}

%Tensor notation undertildes
\usepackage{stackengine} 
\stackMath 
\newcommand\tenq[2][1]{ \def\useanchorwidth{T} \ifnum#1>1 \stackunder[0pt]{\tenq[\numexpr#1-1\relax]{#2}}{\scriptscriptstyle\sim} \else \stackunder[1pt]{#2}{\scriptscriptstyle\sim} \fi } 



\numberwithin{equation}{chapter}


\usepackage{empheq}
 
% Command "alignedbox{}{}" for a box within an align environment
% Source: http://www.latex-community.org/forum/viewtopic.php?f=46&t=8144
\newlength\dlf  % Define a new measure, dlf
\newcommand\alignedbox[2]{
% Argument #1 = before & if there were no box (lhs)
% Argument #2 = after & if there were no box (rhs)
&  % Alignment sign of the line
{
\settowidth\dlf{$\displaystyle #1$}  
    % The width of \dlf is the width of the lhs, with a displaystyle font
\addtolength\dlf{\fboxsep+\fboxrule}  
    % Add to it the distance to the box, and the width of the line of the box
\hspace{-\dlf}  
    % Move everything dlf units to the left, so that & #1 #2 is aligned under #1 & #2
\boxed{#1 #2}
    % Put a box around lhs and rhs
}
}


\begin{document}
\maketitle

\chapter{Introduction}
This document compiles several types of closed form verification tests that can be compared against in the finite element solutions. It provides several examples and the closed form solutions for these examples. 

\chapter{Example 1: Uniaxial Stress on Bar}
This example performs a simple uniaxial stress test on an axisymmetric bar. An example of a mesh that could be applied to this problem 

\section{Closed form solution}

\chapter{Example 2: Pressure Applied to Simply Supported Circular Plate}


\section{Closed form solution}
The equation for the displacement of the center of the plate is as follows:

\begin{equation}
\delta=
\frac{3F(1-\nu^2)D_L^2}{8\pi Eh^3}
\left(
\frac{D_S^2}{D_L^2}\left[1+
\frac{(1-\nu)(D_sS^2-D_L^2)}{2(1+\nu)D^2}\right]
-\left(1+\ln\frac{D_s}{D_L}\right)\right)
\end{equation}

Vitmar, F. F., and Pukh, V. P., ``Method of Determining Sheet Glass
Strength," Zavodskava Laboratoriya, Vol. 29, No. 7, 1963, pp.
863-867.


\chapter{Example 3: Thick walled pressure vessel}

\section{Closed form solution}
The radial displacement of a thick walled pressure vessel at radius $r$ is:
\begin{equation}
u(r)=\frac{1-\nu}{E}
\frac{(r_i^2p_i-r_o^2p_o)r}{r_o^2-r_i^2}+
\frac{1+\nu}{E}
\frac{(p_i-p_o)r_i^2r_o^2}{(r_o^2-r_i^2)r}
\end{equation}

\end{document}